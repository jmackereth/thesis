\chapter{Introduction}
In the study of cosmology -- the process of understanding how the Universe came to be and what that Universe looks like today -- galaxies are perhaps one of the most important observational tracers, with which we can map and measure the makeup and structure of our Universe throughout cosmic time. As a result, robust models for the formation and evolution of galaxies are an essential aspect of the framework by which we might build a good understanding of our Universe. These collections of gas, stars and ambiguous dark matter are the direct result of the complex interplay between cosmology, gravitation and electrodynamic interactions that define the Universe which we live in, and so their characteristics are immutably tied to the very nature of our existence.

Of all the galaxies in our Universe, the one to which we have the closest access to understand the nuanced aspects of its formation and evolution is the galaxy within which we reside: the Milky Way -- \emph{the} Galaxy. The Milky Way presents the problem of galaxy formation at high fidelity, allowing us to test models for its conception and evolution on a star-by-star basis. The assumption that our Galaxy is typical for its mass and other characteristics then allows us to extrapolate these models, trained on the Milky Way, to our understanding of galaxy evolution in general. The overarching goal of this thesis is to test that assumption, asking the question: \emph{is the Milky Way typical?}

By asking this question, we gain a means of properly calibrating any extrapolation of galaxy formation models. If the Milky Way \emph{is} a typical spiral galaxy, whose history is representative of the majority of galaxies at its mass and morphology, then it is a robust frame on which to develop models. If, however, the Milky Way is somehow \emph{atypical}, for example in its size, assembly history, stellar populations, dark matter content, or any other characteristic --  or combination of the above -- then it is important that the use of our detailed knowledge of the Milky Way should be tempered somehow to allow for these `atypicalities'. 

The means to answer the overarching question of the typicality of the Galaxy can only be established through a detailed understanding the place of our Galaxy in the Universe and its galaxy population. In this thesis we intend to make progress towards such an understanding by:
\begin{itemize}
    \item Developing a detailed and state-of-the-art picture of the present day structure and state of the Milky Way, to act as a stringent constraint on present and future models for its formation.
    \item Understanding how galaxies with similar characteristics to the Milky Way emerged, and how these galaxies compare to the mean galaxy population, through analysis of numerical simulations that accurately reproduce the broad properties of the galaxy population.
    \item Testing the predictions of cosmological simulations by reconstructing aspects of the assembly history of the Galaxy, using the most up-to-date data on the Milky Way.
\end{itemize}
The achievement of these more concentrated goals will allow for a re-assessment of the typicality of the Galaxy, and will inform the future interpretation of models for the formation and evolution of the Milky Way in the context of external galaxies.

\section{Understanding galaxies and \emph{the} Galaxy}
In the current popular interpretation of modern cosmology, galaxies are the eventual products of small scale fluctuations in the density of the very early Universe, imprinted upon it by a rapid period of `inflation' very shortly after the Big Bang. It is assumed that dark matter, interacting only with the observable matter in the Universe through gravitation, is the first state of matter to decouple from the 