\chapter{Introduction}
In the study of cosmology -- the process of understanding how the Universe came to be and what that Universe looks like today -- galaxies are perhaps one of the most important observational tracers, with which we can map and measure the makeup and structure of our Universe throughout cosmic time. As a result, robust models for the formation and evolution of galaxies are an essential aspect of the framework by which we might build a good understanding of our Universe. These collections of gas, stars and ambiguous dark matter are the direct result of the complex interplay between cosmology, gravitation and electrodynamic interactions that define the Universe which we live in, and so their characteristics are immutably tied to the very nature of our existence.

Of all the galaxies in our Universe, the one to which we have the closest access to understand the nuanced aspects of its formation and evolution is the galaxy within which we reside: the Milky Way -- \emph{the} Galaxy. The Milky Way presents the problem of galaxy formation at high fidelity, allowing us to test models for its genesis and evolution on a star-by-star basis. The assumption that our Galaxy is typical for its mass and other characteristics then allows us to extrapolate these models, trained on the Milky Way, to our understanding of galaxy evolution in general. The overarching goal of this thesis is to test that assumption, asking the question: \emph{is the Milky Way typical?}

By asking this question, we gain a means of properly calibrating any extrapolation of galaxy formation models. If the Milky Way \emph{is} a typical spiral galaxy, whose history is representative of the majority of galaxies at its mass and morphology, then it is a robust frame on which to develop models. If, however, the Milky Way is somehow \emph{atypical}, for example in its size, assembly history, stellar populations, dark matter content, or any other characteristic --  or combination of the above -- then it is important that the use of our detailed knowledge of the Milky Way should be tempered somehow to allow for these `atypicalities'. 

The means to answer the overarching question of the typicality of the Galaxy can only be established through a detailed understanding the place of our Galaxy in the Universe and its galaxy population. In this thesis we intend to make progress towards such an understanding by:
\begin{itemize}
    \item Developing a detailed and state-of-the-art picture of the present day structure and state of the Milky Way, to act as a stringent constraint on present and future models for its formation.
    \item Understanding how galaxies with similar characteristics to the Milky Way emerged, and how these galaxies compare to the mean galaxy population, through analysis of numerical simulations that accurately reproduce the broad properties of the galaxy population.
    \item Testing the predictions of cosmological simulations by reconstructing aspects of the assembly history of the Galaxy, using the most up-to-date data on the Milky Way.
\end{itemize}
The achievement of these more concentrated goals will allow for a re-assessment of the typicality of the Galaxy, and will inform the future interpretation of models for the formation and evolution of the Milky Way in the context of external galaxies.

\section{Understanding galaxies and \emph{the} Galaxy in the cosmological context}
In the current popular interpretation of modern cosmology, the cold dark matter model \citep[CDM, e.g.][]{1978MNRAS.183..341W}, galaxies are the eventual products of small scale fluctuations in the density of the very early Universe, imprinted upon it by a rapid period of `inflation' very shortly after the Big Bang \citep{guth1981inflationary}. These `seed' fluctuations then grow under their own gravity. Dark matter, thought to interact with baryonic matter only through gravitation, fills these overdensities first, as it is not supported by the radiation which bakes the early Universe. As the density fluctuations grow with infalling dark matter, the baryonic matter slowly begins to cool and collapse into the overdensities, forming the filamentary structure referred to as the `cosmic web'. The gas eventually collapsed deep into the potential wells of the dark matter, forming galaxies of great morphological variety - one of which became \emph{the} Galaxy: the Milky Way.

This simplified picture of structure formation followed by the genesis of galaxies demonstrates how the galaxy population of the Universe is the (indirect) result of cosmology. Na\"ively then, it seems like one could `reverse engineer' the galaxy population, and say something about the nature of our Universe, and in some sense, this is the overarching goal of studies of galaxy formation and evolution. However, this ideal is somewhat complicated by the non-linearity of the processes on galactic scales. The non-linearity of collapse and hierarchical build up of galaxies following the growth of the initial density fluctuations can only be properly realised through direct numerical simulation \citep[e.g.][]{2005Natur.435..629S}. I will argue in the course of this thesis that such simulations, which simulate the above processes self-consistently, are perhaps the most robust means of predicting the context of our Galaxy within CDM and other models.

The study of the galaxy population has a long and colourful history. Of the many early realisations of the variety among galaxies in the Universe, and the perhaps more fundamental realisation of galaxies as external entities to the Milky Way, that which has perhaps best survived to present-day reference is the early work of \citet{1926ApJ....64..321H}, culminating in the widely known Hubble `Tuning-fork' diagram \citep[e.g.][]{1936rene.book.....H,1961hag..book.....S}. This diagram represents a useful summary of galaxy morphologies, but falls short of expressing the true variety of galaxies in our Universe, which display a broad spectrum of features such as bars, bulges, disks, rings, spirals, clumps, flocullence, fountains, streams and shells; all of which are the result of the myriad of processes that shape the galaxy population. Galactic (lower-case and capital G) astrophysics attempts to eke out the physics behind these structures and their connection to galaxy evolution theory and eventually, to cosmology.

While direct constraints on the separate physical origins of all the features of the galaxy population are clearly out of reach, `broad-brush' approaches, such as correlating certain properties of galaxies with one another, lend some insight into the problem of galaxy evolution. For example, it is well established that galaxy morphologies correlate with their colours \citep[e.g.][]{2001AJ....122.1861S}, where blue (likely star-forming) galaxies tend to be those dominated by spiral and disk structures, and red (and therefore presumably non-star forming) galaxies are generally spheroidal in morphology. However, the Galaxy Zoo project \citep{2008MNRAS.389.1179L,2011MNRAS.410..166L}, which allowed for robust, citizen science driven, visual classification of an unprecedented sample of galaxies has shown that there are notable exceptions to this general trend \citep[ e.g.][]{2009MNRAS.396..818S,2010MNRAS.405..783M}. Furthermore, galaxies seem to mainly fall into either of these camps, and very few reside in the intermediate `green' valley; galaxy colours are roughly bimodal \citep[e.g.][]{2004ApJ...600..681B,2006MNRAS.373..469B}. Such a colour bimodality suggests that if transitions between these groups occur, they occur very quickly. \citet{2006MNRAS.373..469B} showed that the fraction of galaxies in each group changes as a function of the number of a galaxy's projected neighbours, suggesting environment somehow plays a role in quenching star formation in the red, dead galaxies. \citet{2010ApJ...721..193P} separated the effects of environment and galaxy mass, demonstrating that star formation quenching is driven by both of these factors separately at different rates, where the environment tends to drive quenching mainly in satellite galaxies, and is independent of the mass of the haloes in which they reside \citep{2012ApJ...757....4P}. \citet{2013MNRAS.428.3306W} demonstrated that distance from the cluster center is a clearer defining factor in understanding satellite quenching, reconciling the data with a dependence on halo mass. Quenching which is dependent on halo mass can be neatly framed as resulting from virial shock heating in the most massive haloes, which shuts down accretion by bringing gas cooling timescales above the dynamical time of the haloes \citep[e.g.][]{2006MNRAS.368....2D,2015MNRAS.447..374G}. Therefore, it seems likely that galaxy colours may indeed have some dependence on cosmology, which must play a role in the halo mass function in the universe, through structure formation.