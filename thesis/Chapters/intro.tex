\chapter{Introduction}
In the study of cosmology -- the process of understanding how the Universe came to be and what that Universe looks like today -- galaxies are perhaps one of the most important observational tracers, with which we can map and measure the makeup and structure of our Universe throughout cosmic time. As a result, robust models for the formation and evolution of galaxies are an essential aspect of the framework by which we might build a good understanding of our Universe. These collections of gas, stars and ambiguous dark matter are the direct result of the complex interplay between cosmology, gravitation and electromagnetic interactions that define the Universe which we live in, and so their characteristics are immutably tied to the very nature of our existence.

Of all the galaxies in our Universe, the one to which we have the closest access to understand the nuanced aspects of its formation and evolution is the galaxy within which we reside: the Milky Way -- \emph{the} Galaxy. The Milky Way presents the problem of galaxy formation at high fidelity, allowing us to test models for its genesis and evolution on a star-by-star basis. The assumption that our Galaxy is typical for its mass and other characteristics then allows us to extrapolate these models, trained on the Milky Way, to our understanding of galaxy evolution in general. The overarching goal of this thesis is to test that assumption, asking the question: \emph{is the Milky Way typical?}

By asking this question, we gain a means of properly calibrating any extrapolation of galaxy formation models, and will likely elucidate new aspects of the formation and history of the Milky Way through answering it. If the Milky Way \emph{is} a typical spiral galaxy, whose history is representative of the majority of galaxies at its mass and morphology, then it is a robust frame on which to develop models. If, however, the Milky Way is somehow \emph{atypical}, for example in its size, assembly history, stellar populations, dark matter content, or any other characteristic --  or combination of the above -- then it is important that the use of our detailed knowledge of the Milky Way should be tempered somehow to allow for these `atypicalities'. Of course, a detailed knowledge of the Milky Way only strengthens our understanding of its individual history of formation and assembly.

The means to answer the overarching question of the typicality of the Galaxy can only be established through a detailed understanding the place of our Galaxy in the Universe and its galaxy population. In this thesis we intend to make progress towards such an understanding by:
\begin{itemize}
    \item Developing a detailed and state-of-the-art picture of the present day structure and state of the Milky Way, to act as a stringent constraint on present and future models for its formation.
    \item Understanding how galaxies with similar characteristics to the Milky Way emerged, and how these galaxies compare to the mean galaxy population, through analysis of numerical simulations that accurately reproduce the broad properties of the galaxy population.
    \item Testing the predictions of cosmological simulations by reconstructing aspects of the assembly history of the Galaxy, using the most up-to-date data on the Milky Way.
\end{itemize}
The achievement of these more concentrated goals will allow for a re-assessment of the typicality of the Galaxy, and will inform the future interpretation of models for the formation and evolution of the Milky Way in the context of external galaxies.

\section{Galaxies in the cosmological context}

Before considering the state of our knowledge of the formation and evolution of the Milky Way, it is essential to set the scene of our current understanding of galaxies in general, and the connection of that understanding to the nature of the Universe. To this end, I briefly touch upon here some important aspects of galaxy formation theory in the cosmological context, with the aim of placing studies of the Milky Way into their much wider context.

\subsection{Cosmological genesis}

In the current popular interpretation of modern cosmology, the cold dark matter model \citep[CDM, e.g.][]{1978MNRAS.183..341W}, galaxies are the eventual products of small scale fluctuations in the density field of the very early Universe, imprinted upon it by a rapid period of `inflation' very shortly after the Big Bang \citep{guth1981inflationary}. These `seed' fluctuations then collapsed under their own gravity as they became significant over the background density, which decreased with the expansion of the Universe. Dark matter, thought to interact with baryonic matter only through gravitation, filled these overdensities first, as it was not supported by the radiation which bakes the early Universe. As the density fluctuations grew with infalling dark matter, the baryonic matter slowly began to cool and collapse into the overdensities, forming the filamentary structure referred to as the `cosmic web'. The gas eventually collapsed deep into the potential wells of the dark matter, forming galaxies of great morphological variety - one of which became \emph{the} Galaxy: the Milky Way.

This simplified picture of structure formation followed by the genesis of galaxies demonstrates how the galaxy population of the Universe is the (indirect) result of cosmology. Na\"ively then, it seems like one could `reverse engineer' the galaxy population, and say something about the nature of our Universe, and in some sense, this is the overarching goal of studies of galaxy formation and evolution. However, this ideal is somewhat complicated by the non-linearity of the processes on galactic scales. The non-linearity of collapse and hierarchical build up of galaxies following the growth of the initial density fluctuations can only be properly realised through direct numerical simulation \citep[e.g.][]{2005Natur.435..629S}. I will argue in the course of this thesis that such simulations, which simulate the above processes self-consistently, and accurately model the hydro-dynamical processes that shape galaxies, are perhaps the most robust means of predicting the context of our Galaxy within CDM and other models.

\subsection{The galaxy-halo connection}
\label{sec:galaxyhaloconnect}
The study of the galaxy population has a long and colourful history. Of the many early realisations of the variety among galaxies in the Universe, and the perhaps more fundamental realisation of galaxies as external entities to the Milky Way, that which has perhaps best survived to present-day reference is the early work of \citet{1926ApJ....64..321H}, culminating in the widely known Hubble `Tuning-fork' diagram \citep[e.g.][]{1936rene.book.....H,1961hag..book.....S}. This diagram represents a useful summary of galaxy morphologies, but falls short of expressing the true variety of galaxies in our Universe, which display a broad spectrum of features such as bars, bulges, disks, rings, spirals, clumps, flocullence, fountains, streams and shells; all of which are the result of the myriad of processes that shape the galaxy population. Galactic (lower-case and capital G) astrophysics attempts to eke out the physics behind these structures and their connection to galaxy evolution theory and thus to cosmology and constraints on the currently accepted cosmological models. 

While direct constraints on the separate physical origins of all the features of the galaxy population are clearly out of easy reach, `broad-brush' approaches, such as correlating certain properties of galaxies with one another, lend some insight into the problem of galaxy evolution. For example, it is well established that galaxy morphologies correlate with their colours \citep[e.g.][]{2001AJ....122.1861S}, where blue (likely star-forming) galaxies tend to be those dominated by spiral and disk structures, and red (and therefore presumably non-star forming) galaxies are generally spheroidal in morphology. However, the Galaxy Zoo project \citep{2008MNRAS.389.1179L,2011MNRAS.410..166L}, which allowed for robust, citizen science driven, visual classification of an unprecedented sample of galaxies has shown that there are notable exceptions to this general trend \citep[ e.g.][]{2009MNRAS.396..818S,2010MNRAS.405..783M}. Furthermore, galaxies mainly fall into either of these camps, and very few reside in the intermediate `green' valley; galaxy colours are roughly bimodal \citep[e.g.][]{2004ApJ...600..681B,2006MNRAS.373..469B}. Such a colour bimodality suggests that if transitions between these groups occur, they occur very quickly. \citet{2006MNRAS.373..469B} showed that the fraction of galaxies in each group changes as a function of the number of a galaxy's projected neighbours, suggesting environment somehow plays a role in quenching star formation in the red, dead galaxies. \citet{2010ApJ...721..193P} separated the effects of environment and galaxy mass, demonstrating that star formation quenching is driven by both of these factors separately at different rates, where the environment tends to drive quenching mainly in satellite galaxies, and appeared to be independent of the mass of the haloes in which they reside \citep{2012ApJ...757....4P}. \citet{2013MNRAS.428.3306W} demonstrated that distance from the cluster center is a clearer defining factor of satellite quenching, and were able to reconcile the data with a dependence on halo mass. Quenching which is dependent on halo mass can be neatly framed as resulting from virial shock heating in the most massive haloes, which shuts down accretion by bringing gas cooling timescales above the dynamical time of the haloes \citep[e.g.][]{2006MNRAS.368....2D,2015MNRAS.447..374G}. Therefore, it may be that through the connection between their halo mass and colour, the galaxies as they appear to us are in some way a product of cosmology, via its dictation of the formation of structure. These results build up to a useful statement on the driving forces behind galaxy evolution, but are only one component of a comprehensive picture of the physics that forms galaxies in the variety we see.

A good example of an agent of galaxy evolution that is seemingly not \emph{directly} driven by cosmology, but is extremely important in reproducing the smaller scale ($\lesssim 0.01$ Mpc) structure in the Universe, is the effect of feedback. Feedback in galaxy formation refers to the regulation of star formation, usually via the attenuation of gas inflow by some process. Generally, feedback is divided into that from stellar sources, such as energy input from supernovae (SNe) or stellar winds, or that from black holes in the central regions of galaxies, e.g. active galactic nuclei (AGN). Allowing gas to collapse into galaxies without feedback means star formation proceeds at rates far higher than that observed, and reach stellar masses tens of times greater than those seen today. However, the inclusion of even relatively simple models for stellar feedback in early simulations and semi-analytic models brings galaxy masses closer inline with observations at the low mass end  \citep[$M\lesssim 10^{12}\ \mathrm{M_\odot}$ e.g.][]{1996ApJS..105...19K,1999MNRAS.310.1087S,2003MNRAS.339..312S}, whereas AGN feedback is necessary to suppress star formation in more massive galaxies \citep[$M\gtrsim 10^{12}\ \mathrm{M_\odot}$][]{2006MNRAS.370..645B,2008MNRAS.391..481S}, where energy injection from SNe is insufficient to pause cold flows. As a result of these feedback processes on both ends of the halo mass scale, the efficiency of star formation peaks in haloes of mass $\sim 10^{12}\ \mathrm{M_\odot}$ \citep[e.g.][]{2013ApJ...770...57B}, which as we will discuss later, is roughly the halo mass of the Milky Way. 

Not only is feedback thought to be important for the regulation of galaxy stellar masses, but likely also plays some role in eventual galaxy morphologies, in particular that of disk galaxies. Without sufficient feedback, low angular momentum material is not removed from central regions of simulated galaxies, which then do not reproduce properties of observed disks \citep[e.g.][and references therein]{2010MNRAS.408..812S}. However, including models for feedback that can efficiently eject such material to be re-accreted at later times with a higher angular momentum produces far more realistic disks, simulated self consistently in a cosmological context \citep[e.g.][]{2011MNRAS.415.1051B,2012MNRAS.427..379M,2013MNRAS.428..129S}. Therefore, the disk morphology of galaxies like the Milky Way is likely folded directly in with its history of feedback. Of course, stellar feedback is a direct result of the star formation history in any galaxy, which is dictated by the supply of gas at a given time, as we have discussed. If, as we touched on above, the gas supply to haloes hosting galaxies is dictated by their mass (to first order), then we arrive back at the large scale structure of the Universe and its cosmology. It is clear then that understanding the star formation and assembly of disk galaxies in detail, and the link to that offered by studying their morphology and stellar content, may provide important insights and constraints on models at even cosmological scales.


\section{The Milky Way as a disk galaxy}

% Having established the place of disk galaxies in our efforts to understand the Universe, and the importance of a detailed understanding of their formation and evolution, I now move to focus solely on that aspect of galactic astrophysics. I will summarise the current understanding of the Milky Way as a disk galaxy, before briefly describing the established model of galactic disk formation. In particular, I aim to introduce the extensive literature on the Milky Way from the last century, and explore the origin and development of near-field cosmology and Galactic archaeology -- the study of the Milky Way as a fossil remnant of the process of galactic formation and evolution (perhaps more accurately expressed as Galactic paleontology!).

Having established the importance of understanding the formation and evolution of disk galaxies in our efforts to understand the Universe, I now move to focus solely on this aspect of galactic astrophysics. Before discussing the progress thus far in understanding the history of formation and evolution of the Milky Way and other disk galaxies, I will first summarise what we know about the Milky Way as a galaxy. 

That we reside in a disk of stars was first postulated by the philosopher Immanuel Kant in 1755, shortly before William Herschel first mapped the shape of the Milky Way through star counting in 1785 \citep{Herschel01011785}. The fact that our Galaxy is just one of the many galaxies in the Universe was not properly understood until the early 20\textsuperscript{th} century, with the work of \citet{1929ApJ....69..103H}. Since that early work, we have come to be able to properly characterise our Galaxy and compare it to the others we observe, although as we will discuss, much of this knowledge is now being tested when confronted with the latest data.

The most obvious way by which to compare the Milky Way to its extra-galactic counterparts is through its morphology and structural parameters relating to its size. Such properties are readily measured for external galaxies and have been for some time \citep[e.g.][]{1959HDP....53..311D}. In terms of morphology, it is well established that the Milky Way has an extended stellar disk with spiral arms, a bulge component, a bar and a diffuse stellar halo with evidence of substructure. However, the exact details of these features are difficult to discern owing to the fact that much of the Galaxy is obscured by extinction as a result of our position in the disk.

\subsection{The Disk}

\subsubsection{Spiral Structure}
The feature of our galaxy which can be compared to other disks with almost complete ubiquity is its spiral structure. However, our knowledge of the spiral structure is rather contentious. Mapping of the spiral structure is commonly performed in the near infra-red (NIR) and radio regime, overcoming much of the extinction to look at star forming regions of the Galaxy. Early studies of HII regions defined the `standard model' for the spiral structure of our Galaxy as a four armed spiral close to an Sc type with a $12^{\circ}$ pitch angle  \citep{1976A&A....49...57G}. More recent studies have have struggled to agree on the number of arms between 2 and 4 \citep[e.g.][]{1976A&A....46..261S,1980ApJ...239L..53C,1981ApJ...250..551B,1995ApJ...454..119V,2000A&A...358L..13D,2003A&A...397..133R}, generally because of the difficulties in measuring distances. Very Long Baseline Interferometry (VLBI) meaurements, which allow direct and precise trigonometric parallax and proper motion measurements favour a four armed spiral structure \citep{2009ApJ...700..137R,2014ApJ...783..130R}.

The nature of the spiral arms is the subject of much debate, splitting into two main streams between long lived stationary density wave models \citep[e.g.][]{1964ApJ...140..646L,1996ssgd.book.....B}, and models where the arms are transient and recurrent through the history of the disk \citep[e.g.][]{1981seng.proc..111T,1984ApJ...282...61S}. It is well established that tidal forces from satellites can give rise to transient spirals \citep[e.g.][]{2010MNRAS.403..625D}, but spiral patterns do arise in isolated simulations of thin disks. Indeed, the lifetime of the arms appears to be one of the more important questions to ask, which would discrimate between these models \citep{2011MNRAS.410.1637S}. In a recent example of how such a discrimination might be made, \citet{2018MNRAS.481.3794H} modelled the velocity field in the solar neighbourhood under the effects of transient winding spiral arm perturbations and found good agreement with \emph{Gaia} DR2 data. 

\subsubsection{Size and Thickness}
The vertical extent of the disk is relatively well characterised at the solar radius due to the relative lack of extinction at high Galactic latititudes. For a long time, the disk has been divided into a thick and thin component, owing to the early results of \citet{1983MNRAS.202.1025G}, and the knowledge that such structures appear to exist in external disk galaxies \citep[e.g.][]{1979ApJ...234..829B,1979ApJ...234..842T,2006AJ....131..226Y}. Of many different measurements, a good baseline for the characteristic heights of these disk components is found in the tomography of M dwarfs from SDSS data in \citet{2008ApJ...673..864J}, who find exponential scale heights of 300 and 900 pc for the thin and thick component, respectively. In this sense, the Milky Way sits relatively well with external galaxy measurements, which have similar heights and also have radially extended thick components \citep[e.g][]{2006AJ....131..226Y}. However, the simple view of these disk components as geometric entities breaks down somewhat when the disk is dissected further.

A more natural separation can be made in the disk in its $\alpha$ element abundances (which we discuss further in Section \ref{sec:alpha}), whereby the disk can be divided into a high and low \afe{} component \citep[e.g.][]{1998A&A...338..161F,2003A&A...410..527B,2005A&A...433..185B,2013A&A...560A.109H,2014A&A...562A..71B,2014A&A...564A.115A,2014ApJ...796...38N,2015ApJ...808..132H}. These components are found to have similarities to the geometrically divided thin/thick disk, such that the high \afe{} disk has scale height $\sim 1$ kpc, and the low \afe{} disk has a scale height as low as $\sim 0.2$ kpc \citep[e.g.][]{2012ApJ...753..148B,2016ApJ...823...30B}. However, when defined in this way, the high \afe{} component has a relatively short scale length at $\sim 2$ kpc, compared to the geometrically defined thick disk which has a scale length closer to $\sim4$ kpc \citep{2008ApJ...673..864J}. The low \afe{} component has a flatter radial profile \citep{2012ApJ...752...51C}, which has a complex density structure resembling donut shaped annuli, such that the density increases out to a break radius, and decreases outside this radius \citep{2012ApJ...753..148B,2016ApJ...823...30B}. Furthermore, when the disk is divided into sub-populations according to the \afe{} and \feh{} abundances of its stars, it seems that the vertical mass distribution is smooth, suggesting that the `thick' disk may not be so distinct \citep{2012ApJ...751..131B}. Combining all the sup-populations, this can be reconciled with the double exponential found in the star counts by \citet{1983MNRAS.202.1025G} \citep[see][]{2013A&ARv..21...61R}. The vertical kinematics of the disk as a function of \afe{} are consistent with a smooth vertical structure \citep{2012ApJ...755..115B}, and it is clear that the high \afe{} disk is more kinematically hot than the low \afe{} stars \citep[e.g.][]{2005A&A...433..185B}. It is clear, therefore, that the link between the disk structure and its element abundances may be able to offer significant insight into the formation of the galaxy.

\subsubsection{Element Abundances}

The disentangling of the origins of element abundance patterns in the solar vicinity and beyond offers many constraints on models for the formation of the Galaxy \citep[a seminal review on the goals of this effort is given by][]{2002ARA&A..40..487F}. A good first order tracer of the history of star formation in the Galactic disk is offered by the metallicity of stars, as measured by their \feh{} abundance. Commonly, models try to predict the \feh{} gradient with Galactocentric radius, the metallicity gradient, which can be readily compared to external galaxies. Metallicity gradients are measured by a variety of tracers by which the variation in the gradient with age can also be measured, including planetary nebulae \citep[e.g.][]{1994A&A...282..436M,1994Ap&SS.219..231M,2010ApJ...714.1096S,2011ApJ...738...27B}, open clusters \citep[e.g.][]{1998MNRAS.296.1045C,2002AJ....124.2693F,2004A&A...414..163S,2009A&A...494...95M,2016AN....337..922C} and field stars \citep[e.g][]{2004A&A...418..989N,2014A&A...566A..37G}. Among these studies, the trends between the \feh{} gradient and the tracer age seem to vary. This was recently studied using precise age estimates in field stars and comparison with a mock catalogue based on a numerical simulation, showing that strong radial migration and age systematics may account for some of the discrepanices \citet{2016arXiv160804951A}.



\section{Galactic archaeology: fossilised galaxy formation}

\subsection{A schematic for the formation of disk galaxies}

A useful way to present the wealth of work on the formation of the Milky Way is to describe a schematic of its formation as a disk galaxy, as unveiled by studies of it throughout the last century. The conception of the `canonical' model for the birth of the Milky Way in a rapid collapse of a proto-galactic gas cloud \citep[the ELS model;][]{1962ApJ...136..748E} makes a good starting point for such a schematic. 

In what can be considered as the first implementation of Galactic archaeology, \citet{1962ApJ...136..748E} showed that the orbital energies and eccentricities of stars with lower metallicities were increased, inferring that these stars were members of the first generation of stars formed from the originally rapidly collapsed protocloud of our galaxy. The later work of \citet{1978ApJ...225..357S}, which inferred that the varying element abundances of Galactic globular cluster (GC) systems, suggested a different view of the formation of the galaxy through the conglomeration of `protogalactic fragments'. These seminal works underpin many of the modern studies of the Milky Way, and are commonly framed as competitive scenarios. However, at the time, these findings mirrored the developments in the CDM model \citep{1978MNRAS.183..341W}, and analytical models of the collapse of gas in a halo which gained angular momentum from hierarchical clustering produced seemingly realistic disks \citep{1980MNRAS.193..189F}. While analytic approaches would work well at describing galaxy disk formation, hydrodynamical simulations in a cosmological context struggled to produce realistic populations of disk galaxies until advances in the inclusion of detailed feedback models, as expressed in Section \ref{sec:galaxyhaloconnect}. 






