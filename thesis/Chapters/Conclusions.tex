\chapter{Summary and Conclusions}
\label{chapter:conclusions}
In the preceding chapters, I have presented novel results that provide insight into the history of the Milky Way and the nature of it. In this chapter I will attempt to combine these findings into a coherent new picture of the formation of the Milky Way, drawing also (where necessary) from the body of literature which I discussed in Chapter \ref{chapter:intro} and the introductions and concluding sections of each of the other chapters. These new insights into the evolution and resulting structure of the Galaxy will allow for a re-assessment of the question of typicality of the Milky Way, inline with the overarching goal of this thesis. 

Acting as a baseline constraint on new models for the formation of the Galaxy, Chapter \ref{chapter:apogeestruc} presented an up to date mapping of the disc of the Galaxy as a function of age, \feh{} and \afe{}. It showed that the vertical structure of the disc appears to be smooth in terms of the stellar mass contribution as a function of scale height, with no clear distinction between the high and low-\afe{} populations in terms of vertical structure, confirming the same findings from SEGUE of \citet{2012ApJ...751..131B}. However, I showed that the low and high-\afe{} populations have a distinct \emph{radial} structure which was hitherto unseen by SEGUE, but noted by \citet{2016ApJ...823...30B} using APOGEE data also, reviving the notion that these populations may in fact be structurally distinct, but only in the sense that the high-\afe{} population is centrally concentrated, and \emph{not} distinct in a vertical sense at the solar radius - presumably by coincidence of our position in the Galaxy. 

Chapter \ref{chapter:eagle} presented a new model for the formation of bimodality in \afe{} at fixed \feh{} in galaxy discs in the EAGLE simulations. EAGLE suggests that high-\afe{} stars can only be readily formed in very high density environments, where the gas consumption timescale (linked to the star formation efficiency) becomes shorter than the characteristic timescale for SN Type Ia enrichment. In order to develop a chemically distinct high-\afe{} population, galaxies must undergo an early period of rapid accretion, meaning that the haloes hosting bimodal galaxies are those with the most rapid dark matter accretion between $\tau \sim 4$ to $6$ Gyr. Because of this, galaxies with \afe{} bimodality are rare in the simulation, making up only $\sim 6\%$ of Milky Way mass galaxies. This strongly suggests that the Milky Way had an atypical assembly history, which lead it to become a very atypical disc galaxy at it's stellar mass. The simulations predict that the Milky Way then should have a very centrally concentrated high-\afe{} population, formed in a rapid collapse at early times, leaving a kinematic and structural distinction between it and the low-\afe{} population, which is more extended and formed at late times as a cool disc. This appears to be the case, at least in terms of spatial structure, given the findings of Chapter \ref{chapter:apogeestruc}.

The prediction of an atypical accretion history for galaxies hosting a bimodality in their \afe{} distribution at fixed \feh{} in EAGLE suggests that the Milky Way should also have experienced an irregular history of accretion. I tested this prediction in Chapter \ref{chapter:highe}, demonstrating that the \emph{Gaia} DR2 data, in combination with APOGEE, reveals that the Milky Way stellar halo harbours a hitherto unseen massive accreted component, which becomes clear when orbital properties and chemical abundances are combined. In the EAGLE simulations, these accreted debris populations are extremely rare, and result from the late-time ($z\sim 1$ to $2$) accretion of relatively massive ($10^{8} < M_{*} < 10^{9}\ \mathrm{M_{\odot}}$) systems. The element abundances of the satellite debris in EAGLE agree well with those of the Milky Way debris, and further suggest that the stellar mass of the progenitor must have been in the range of $\sim 10^{8.5}\ \mathrm{M_{\odot}}$. These findings present a first insight into how the predictions of EAGLE and future simulations might be tested in the Milky Way halo.

Returning to the main goals of the thesis as outlined at the 
start of Chapter \ref{chapter:intro}, this thesis has:
\begin{itemize}
    \item Presented a detailed and state-of-the-art snapshot of the structure of the Milky Way as a function of stellar age and element abundances, which provides a more stringent constraint on models for its formation, which generally predict the structure as a function of age.
    \item Shown that cosmological simulations produce galaxies with element abundance patterns consistent with those of the Milky Way, and that these galaxies are \emph{not} typical, at least compared to other disc galaxies in the stellar mass range of the Milky Way. The most fundamental aspect of this atypicality appears to be in their mass assembly histories.
    \item Tested the predictions of the cosmological simulations by reconstructing aspects of the assembly history of the Galaxy using the \emph{Gaia} DR2 and APOGEE data. I also showed that the simulations can correctly predict the element abundances of accreted satellite debris, which highlights their usefulness for future studies of satellite galaxies.
\end{itemize}
From these points of view, it would seem that the Milky Way is certainly \emph{not} typical of a disc galaxy at its stellar mass, and likely had a far more active early history of mass assembly than its similarly massive siblings. This presumably lead to its structure at the present day, harbouring a centrally concentrated high-\afe{} population which was formed in this early episode. The debris in the stellar halo of our Galaxy provides an excellent tool to test this prediction of an early assembly, and the results in Chapter \ref{chapter:highe} are merely a small step toward the full reconstruction of the Milky Way assembly history.

\section{Implications of an atypical Milky Way}

As has been discussed at length, a key result of this thesis is that the Milky Way may be atypical for a disc galaxy at its stellar mass. This finding presents a number of interesting consequences for galaxy evolution theory which I will touch upon here before concluding with some future prospects for the field of Galactic Archaeology. 

Chapter \ref{chapter:eagle} showed that the $\alpha$-element abundances in the Galaxy may be strongly linked to its assembly history, which the EAGLE simulations suggest was atypically rapid at early times relative to other disc galaxies at the same stellar mass. This finding suggests an interesting potential to use the element abundances of the Milky Way to make constraints on its assembly history, and even to constrain with higher fidelity the mass function of its progenitor galaxies. Doing so will require a more detailed understanding of the link between halo assembly and the properties of the galaxies that form in the simulations, and how this shapes their element abundances. Works studying such connections, such as that of \citet{2018arXiv180505956M}, for example, are intriguing first steps toward such a goal. The next section will briefly discuss the requirements of future simulations to achieve these goals.

Also linked to its atypical assembly history, Chapter \ref{chapter:eagle} suggests that the star formation history and, consequently, the age distribution of stars in the Milky Way should indicate an early burst of star formation which formed the high-\afe{} population. Star formation histories based on local samples suggest that the number of old ($\gtrsim 8$ Gyr) stars is suggestive of a rapid formation of the early disc \citep{2015A&A...578A..87S}, but are commensurate with observational expectations of the stellar mass build up of galaxies at the mass of the Milky Way \citep{2014ApJ...781L..31S}. It is worth noting that Chapter \ref{chapter:apogeestruc} presented a selection function corrected age distribution at the solar radius which was peaked for younger stars, in some contention with these results. The stellar age distribution toward the bulge, observational difficulties aside, suggests a very efficient early history of star formation \citep[e.g.][]{2018MNRAS.477.3507B}. Therefore, it is clear that a full reconstruction of the star formation history of the whole Galaxy will require careful attention to be given to survey selection biases. Section \ref{sec:future} will also describe how improvements in stellar age estimates might help this effort.

Further to better measurements of the star formation history of the Milky Way, it may also be possible to test the prediction of atypicality through measurement of the assembly and star formation histories of external galaxies. Exquisite spatially resolved spectroscopy from IFU instruments such as that from the SDSS MaNGA survey \citep[Mapping Nearby Galaxies at Apache Point Observatory;][]{2015ApJ...798....7B} are allowing studies of the star formation and assembly histories of statistical samples of galaxies \citep[e.g.][]{2016MNRAS.463.2799I,2018MNRAS.480.2544R}. Examining these datasets and studying the links between the features of the galaxy population, the Milky Way and the predictions of simulations like EAGLE, may allow for robust tests of the model for the formation of the Galaxy proposed by this thesis.

Alongside the implications discussed above, an atypical assembly history for the Galaxy raises some important questions regarding other known features of the Galaxy which appear atypical. For example, it is known that the central black hole of our Galaxy, at $M \simeq 4\times 10^{6}\ \mathrm{M_{\odot}}$ \citep[e.g.][]{2009ApJ...707L.114G}, brings our Galaxy far off the black hole mass vs. velocity dispersion relationship for classical bulges \citep[e.g.][]{2013ARA&A..51..511K}. The fact that the Milky Way appears to have a small black hole is in apparent contention with the finding here that it may have had an earlier accretion history, as one might expect that the black hole may have built up more mass in this scenario. Understanding this discrepancy may help understand further the prediction of an early accretion history for the Milky Way. Additionally, as expressed in Chapter \ref{chapter:intro}, and confirmed by the work presented in Chapter \ref{chapter:apogeestruc}, our galaxy also appears to have a more compact disc than other galaxies at a similar mass \citep[e.g. those shown in][]{2010MNRAS.406.1595F}, with only the youngest populations extending as far as the mean scale length of those external galaxies. The understanding of how these different features might be linked to each other will provide further insights into the evolution of the Milky Way. Regardless, the mounting evidence seems to suggest that our Galaxy is indeed rather atypical.



\section{Future Prospects}
\label{sec:future}
With the oncoming deluge of data for stars in the Milky Way, not just from the future \emph{Gaia} data releases \citep[which will include epoch spectroscopy and photometry, offering the potential to mine the data for element abundances and ages, e.g.][]{2016A&A...595A...1G}, but also from planned spectroscopic surveys (e.g. MOONS, \citeauthor{2012SPIE.8446E..0SC} \citeyear{2012SPIE.8446E..0SC}; WEAVE, \citeauthor{2012SPIE.8446E..0PD} \citeyear{2012SPIE.8446E..0PD}; and 4MOST, \citeauthor{2016SPIE.9908E..1OD} \citeyear{2016SPIE.9908E..1OD}) prospects are very strong for a more complete insight into the history of the Galaxy. Not only will improved and larger numbers of spectroscopic and astrometric information become available, but there are also a number of ongoing and planned asteroseismic surveys, such as TESS \citep{2015JATIS...1a4003R} and PLATO \citep[e.g.][]{2017AN....338..644M}. Improvements in the data processing and our understanding of asteroseismic data from completed missions like \emph{Kepler} \citep{2010Sci...327..977B}, will also provide better measurements of stellar oscillations. Improved asteroseismology for a large number of stars over wide ranges in parameter space would generate great improvements in our estimates for stellar masses, and therefore ages, allowing for better training sets for spectroscopically estimating ages \citep[Such as those in Chapter \ref{chapter:apogeestruc}, from methods like that of e.g.][]{2016MNRAS.456.3655M}. 

Improvements in age estimates for large stellar samples will provide detailed insights into the history of the Galaxy. Currently, extending spectroscopic age estimates to low metallicity (\feh{}$< -0.5$ dex) is problematic, owing to both problems with asteroseismic models at these metallicities \citep[e.g.][]{2014ApJ...785L..28E} and to the relative lack of stars at low \feh{} which can be used as training data. At present, the best age estimates have uncertainties in the range $\sim 30\%$, blurring the age distributions in the oldest stars by $\gtrsim 4$ Gyr. As most stars in the high-\afe{} disc are older than $\sim 9$ Gyr (as seen in Chapter \ref{chapter:apogeestruc}) and many of its stars have \feh$< -0.5$ dex, this means that the distinction in age between the low and high-\afe{} discs is as yet unclear. Understanding the true age distribution and trends in these populations would be a good test of the model for the \afe{} bimodality presented in Chapter \ref{chapter:eagle}. Additionally, accreted debris such as that discovered in Chapter \ref{chapter:highe} of this thesis are unlikely to have robustly measured ages in the current paradigm. Only good age measurements will be able to place the strongest constraints on the nature and accretion time of this debris. 

With improved age estimates, and a wider (in terms of Galactocentric radius) and more complete coverage of multidimensional abundance information in the disc, bulge and halo we will be better able to reconstruct the history of the Galaxy directly. However, there are also stringent requirements for future theoretical approaches. The usefulness of zoom-in and idealised simulations of the Milky Way is clear from the wide-ranging extant literature, but this thesis makes clear the need for large volume simulations in a cosmological context to provide predictions for the Milky Way and disc galaxies in a wider context. Only by combining such approaches will we be able to properly make robust models for disc galaxy formation \emph{in general}. 

The unprecedented amount and quality of observational data that will come available in the next few years outlines the desiderata for new and improved simulations of galaxy formation. As an example, for making the most powerful predictions for Galactic archaeology, it would be useful to make various improvements in the sub-grid treatment of chemical enrichment, given that many of the inferences made on the formation of the Milky Way revolve around measures of its element abundances. In particular, better use of the knowledge of stellar mass loss and element abundance yields from the stellar astrophysics community might allow for improved agreement between the simulations and data in terms of the chemical evolution of galaxies. Improving these aspects of large volume cosmological simulations would allow for far stronger predictions for the formation of Milky Way like galaxies based on multiple element abundances. While improvements and tests of the currently used methods for chemical enrichment have been made \citep[e.g.][]{2009MNRAS.399..574W}, the implementation of the process into SPH (as used in EAGLE) is not yet perfected.

The idea of making general models for the formation of disc galaxies through Galactic archaeology requires that we understand in what ways our Galaxy has evolved differently to the general population, in as much detail as possible. This thesis has given novel evidence that our Galaxy may indeed have had an atypical history of formation and evolution, caused in part by its accretion history, which is predicted to have diverged from the main population of $L_*$ galaxies at early cosmic time. While this may be somewhat uncomfortable to Galactic `archaeologists' at face-value, an atypicality of the Milky Way will still allow us to probe galaxy evolution and the effect on galaxies that such atypical evolution has. Understanding this feature of our Galaxy well will be essential to our ability to better understand and characterise the formation of disc galaxies.