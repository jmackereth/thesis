\chapter*{Publications}
\addcontentsline{toc}{chapter}{Publications}

During the course of the preparation of this thesis, the work within Chapters \ref{chapter:apogeestruc},  \ref{chapter:eagle} and \ref{chapter:highe} has been presented in the following jointly authored publications:

\begin{itemize}

\item[1] {\it The age-metallicity structure of the Milky Way disc using APOGEE} \\
{\bf Mackereth, J. T.}, Bovy J., Schiavon R. P., Zasowski G., Cunha K., Frinchaboy P. M., Garc\'ia Perez A. E., Hayden M. R., Holtzman J., Majewski S. R., M\'esz\'aros S., Nidever D. L., Pinnsonneault M., Shetrone M. D., 2017, MNRAS, 471, 3057

\item[2] {\it The origin of diverse $\alpha$-element abundances in galaxy discs} \\
{\bf Mackereth, J. T.}, Crain R. A., Schiavon R. P., Schaye J., Theuns T., Schaller M., 2018, MNRAS, 477, 5072

\item[3] {\it The origin of accreted stellar halo populations in the Milky Way using APOGEE, Gaia and the EAGLE simulations} \\
{\bf Mackereth, J. T.},Schiavon R. P., Pfeffer J., Hayes C. R., Bovy J., Anguiano B, Allende Prieto C., Hasselquist S., Holtzman J., Johnson J. A., Majewski S. R., O'Connell R., Shetrone M. D., Tissera P. B., Fernández-Trincado J. G., 2018, MNRAS, 482, 3426


\item[4] {\it Fast estimation of orbital parameters in Milky-Way-like potentials} \\
{\bf Mackereth, J. T.}, Bovy J., 2018, PASP, 130, 114501

\end{itemize}

Publication 1 forms the basis of Chapter \ref{chapter:apogeestruc}. The analysis preseneted therein is based on that previously presented in \citet{2016ApJ...823...30B}, and builds upon the codebase and methodology used in that work. All new analysis and interpretation was conducted by J. T. Mackereth. The initial paper draft was prepared in entirety by J. T. Mackereth, and minor comments from co-authors and an anonymous referee were incorporated at a later date.

Chapter \ref{chapter:eagle} is based on material from Publication 2. The EAGLE simulations used in that chapter and also in Chapter \ref{chapter:highe} were run and described by \citet{2015MNRAS.446..521S,2015MNRAS.450.1937C}. All analysis and interpretation of the simulations was carried out independently by J. T. Mackereth, with intellectual input from R. A. Crain, J. Schaye and R. P. Schiavon. The original manuscript was prepared by J. T. Mackereth, and minor comments from co-authors, colleagues and an anonymous referee incorporated later.

Publication 3 is the basis of the study presented in Chapter \ref{chapter:highe}, and is based on the serendipitous discovery of a potentially accreted stellar halo population in the \emph{Gaia} DR2 data, discovered independently by J. T. Mackereth shortly after the data release. All data analysis and interpretation was carried out by J. T. Mackereth, with significant intellectual input from R.P. Schiavon. A catalogue of accreted satellite particles in EAGLE was prepared by J. Pfeffer for use in the analysis. The manuscript was prepared by J. T. Mackereth, with additions made following comments from co-authors, colleagues and an anonymous referee.

The method for the fast estimation of orbital parameters via the St\"ackel approximation, presented in Publication 4, is used extensively in Chapter \ref{chapter:highe}. While this paper does not form the foundation for any chapters of the thesis, all aspects of that work were developed in close collaboration with J. Bovy.



\vfill
{\sc \AuthorName \hfill\today}