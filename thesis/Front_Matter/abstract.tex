\chapter*{Abstract}
\addcontentsline{toc}{chapter}{Abstract}

The galaxy within which we reside, the Milky Way, offers perhaps the highest fidelity training ground for models of galaxy formation, providing insights into its history of formation and evolution on a star-by-star basis. Such insights are only truly useful for constraining disc galaxy formation models if we properly understand the wider context of the Milky Way among the plethora of galaxies in the Universe. This thesis aims to make progress toward answering the question of whether the Milky Way is a typical disc galaxy. Through the effort to answer this question, this thesis presents new measurements of the structure of our Galaxy and new insights into aspects of its history of assembly and evolution which have a strong influence on its $\alpha$-element abundances. 

Providing a baseline constraint on future models for the formation of the Galaxy, I present a dissection of the Milky Way disc spatial structure as a function of \feh{}, \afe{} and stellar ages (based on the surface abundances of Carbon and Nitrogen), as measured by the APOGEE survey. I measure the disc density profile, fitting the scale heights and lengths of mono-age, mono-\feh{} sub-populations in the high and low \afe{} disc. The fitted disc vertical scale height distribution is smooth when weighted by surface-mass density, suggesting that the high and low \afe{} populations are not vertically distinct at the solar radius, as would be expected if they were interchangeable with the geometric thin and thick disc components. I find that the surface density profile of low \afe{} mono-age, mono-\feh{} populations is best fit by a broken exponential, such that their density increases with $R$ to a peak radius, and declines thereafter, whereas the high \afe{} populations are better described by a single, declining exponential within the range of Galactocentric radii observed by APOGEE. The trends of the density profile parameters as a function of age and metallicity provide insights into the structural evolution of the Galaxy, and provide strong constraints on future models of its formation and evolution. In particular, a main finding of this study is that the high and low \afe{} discs have a relatively distinct \emph{radial} structure.

To understand the origin of the structures found in the above study, and to generate novel and predictive models for the formation of the Galaxy, I perform an analysis of element abundances in Milky Way like galaxy discs in the EAGLE simulation. I concentrate on the abundance of $\alpha$-elements in these galaxies, with a view to understanding the origin of the bimodality in \afe{} at fixed \feh{} which is apparent in both the Milky Way and in EAGLE Galactic analogues. I show that EAGLE reproduces broad expectations for the production of $\alpha$-elements from simple chemical evolution models, namely that stars with enhanced \afe{} are born from gas in rapidly star forming environments at high density. These environments are conducive to forming $\alpha$-enhanced stars because their gas consumption timescale is considerably shorter than the characteristic delay timescale for Type Ia supernovae. I further show that such environments are only achieved in EAGLE haloes when the rate of dark matter accretion is faster at early times than the majority of galaxies with similar stellar mass at $z=0$. The necessity for this atypical and rapid early assembly means that galaxies hosting discs with \afe{} bimodality are extremely rare, forming in $\simeq 6\%$ of galaxies at the Milky Way stellar mass range.

Following the prediction of EAGLE that galaxies which host \afe{} bimodality like that of the Milky Way should have atypical histories of assembly, I then perform a study of Milky Way halo stars in common between APOGEE and \emph{Gaia} DR2, with a view to placing constraints on the accretion history of the Galaxy. I present a detailed characterisation of the kinematics and abundances of the recently discovered \emph{Gaia}-Enceladus association, which is proposed to be the debris of a singular and massive accretion event. By comparing the kinematics and abundances of \emph{Gaia}-Enceladus to the debris of satellites accreted onto Milky Way analogues in the EAGLE simulations, I make a quantitative prediction of the stellar mass of the Milky Way satellite to be $10^{8.5} < M_{*} < 10^{9}\ \mathrm{M_{\odot}}$ and predict its earliest possible time of accretion to be $z\sim1.5$. I also show that such mergers are uncommon in the simulations, in agreement with the prediction that the Milky Way assembly history should be atypical.

The above results outline a new view of the Galaxy in which it is potentially not a good example of a `typical' disc galaxy, playing host to structural components which are linked to its chemistry in a complex manner, which may indicate that its history of assembly is not in common with other galaxies at the same stellar mass. In finding that the Galaxy is atypical in this way, this thesis has uncovered new aspects of the evolutionary history of the Milky Way, which pave the way for future work towards the goal of fully reconstructing the history of our Galaxy and using that understanding to formulate robust and general models for the formation of disc galaxies. 
\vfill
{\sc \AuthorName \hfill\today}